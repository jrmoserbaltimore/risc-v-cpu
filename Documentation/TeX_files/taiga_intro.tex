\chapter{Taiga}

Eric Matthews and Lesley Shannon created Taiga "to facilitate research into heterogeneous processor systems and high performance architectural features for soft processors."  Their goals included high instruction-level paralelization (ILP) at competitive operating frequencies (fMax) and area usage.

In 2017, Taiga ran on a Xilinx Zynq X7CZ020 FPGA at 111MHz in its minimal without multiply, divide, TLB, MMU, cache, or atomic operations, and 103MHz at full configuration.  This version also achieved 254MHz on a Xilinx Virtex UltraScale+ XCVU9P.  The critical path was the write-back data path in all except the full configuration, which was limited by access to data TLB tags.

In 2019, they tested Quick-Div on a Xilinx Virtex UltraScale+ VCU118 FPGA board using an XCVU9P-L2FLGA2014E.  They needed Quick-Div to run at 373MHz to stay out of the critical path—just shy of 50\% higher fMax than they achieved on the XCVU9P in 2017.  Clearly, they have improved the processor over time.

Nearly 400MHz performance, even on a modern FPGA, is impressive.  Taiga achieves clock rates comparable to iDEA, yet implements a full RV32 RISC-V processor capable of running Linux.

