
\begin{figure}
% Kerberos overview, similar to Taiga diagram
    \begin{center}
        \begin{tikzpicture}[scale=0.2]
            \tikzstyle{every node}+=[inner sep=0pt]

            % Global BHR
            \path
                 % Node and label
                 [black,draw,thick]
                 % Origin
                 (0,0) node [black] (Global BHR) {}
                 ++(6,2) node {\small{Global BHR}};
            % Shift register
            \foreach \x in {0,...,11}
                \path [black, draw, thick]
                 (Global BHR) ++(\x,0) -- ++(1,0) -- ++(0,-1) -- ++(-1,0) -- ++(0,1) -- cycle;
             % Grey boxes
             \foreach \x in {1,3,4,7,8,11}
                \path [black, draw, thick, fill=gray]
             (Global BHR) ++(\x,0) -- ++(1,0) -- ++(0,-1) -- ++(-1,0) -- ++(0,1) -- cycle;

            \path
                [black,draw,thick]
                (Global BHR) ++(18,-4) node (PHRT) {}
              ++(6,2) node {\small{PHRT}};
            \path
                [black,draw,thick]
                (PHRT) ++(18,0) node (PT) {}
              ++(2,2) node {\small{PT}};

            \path
                [black,draw,thick]
                (Global BHR) ++(-3,-4) node (Local BHT) {}
                ++(4,-6) node [rotate=90]{\small{Local BHT}};

            % Local BHT
            \path
                [black,draw,thick]
                (Global BHR) ++(10,-4.5) node (Global BHT) {}
              ++(4,-6) node [rotate=90]{\small{Global BHT}};
            \foreach \y in {0,...,11}
                \foreach \x in {0,1,13,14,21,22,...,33,40,41}
                  \path [black, draw, thick]
                  (Local BHT)
                ++($(\x,0) + -1.5*(0,\y)$)
                  -- ++(1,0) -- ++(0,-1) -- ++(-1,0) -- ++(0,1) -- cycle;

%            \foreach \y in {0,...,11}
%                \foreach \x in {14,...,15}
%                \path [black, draw, thick]
%                (Local BHT)
%              ++($(\x,0) + -1.5*(0,\y)$)
%                -- ++(1,0) -- ++(0,-1) -- ++(-1,0) -- ++(0,1) -- cycle;

            %%%%%% All this to make it look random
            \foreach \y in {0,...,11}
                \foreach \x in {0,1,13,14,21,22,...,33,40,41}
                \ifthenelse{
                   \intcalcMod{
                     \intcalcAdd{
                      \x}{
                      \intcalcMul{\y}{7}
                     }
                   }{3}
                > 0 \AND
                \intcalcMod{
                    \intcalcAdd{
                        \y}{
                        \intcalcMul{\x}{11}
                    }
                }{7} > 1
            }{
                  \path [black, draw, thick, fill=gray]
                    (Local BHT)
                ++($(\x,0) + -1.5*(0,\y)$)
                -- ++(1,0) -- ++(0,-1) -- ++(-1,0) -- ++(0,1) -- cycle}{};
            %%%%% Never try to read that ugly code

        \path
            [black,draw,thick]
            (Local BHT) ++(-1,0)
            -- ++(0,-18) -- ++(-2,4) -- ++(0,10) -- ++(2,4) -- cycle
            (Local BHT)
          ++(-8, 6) node {Address};
        \path
            [black,draw,thick]
            (Global BHR) ++(9,-4)
            -- ++(0,-18) -- ++(-2,4) -- ++(0,10) -- ++(2,4) -- cycle;
        \path
            [black,draw,thick]
            (PHRT) ++(14,0)
            -- ++(0,-18) -- ++(2,4) -- ++(0,10) -- ++(-2,4) -- cycle
            (PHRT)
          ++(8,6) node {Address}
            (PT) ++(4,0)
            -- ++(0,-18) -- ++(2,4) -- ++(0,10) -- ++(-2,4) -- cycle;
        %%%% MUX traces
        \path
            [black,draw,thick]
            (Local BHT) ++(-3,-9)
            -- ++(-1,0) -- ++(0,-12)
            (Global BHR) ++(7,-13)
            -- ++(-1,0) -- ++(0,-12)
            (PHRT) ++(16,-9);
        \path
            [black,draw,thick]
            (Local BHT) ++(-4,6)
            -- ++(2,0) -- ++(0,-8);
        \path[black,draw,thick]
            (Global BHR) ++(8,-1)
            -- ++(0,-5);

        \path
            [black,draw,thick]
            (PHRT) ++(16,-9)
            -- ++(1,0) -- ++(0,12) -- ++(6.25,0) -- ++(0,-5.5)
            (PHRT) ++(24,-9)
            -- ++(1,0) -- ++(0,-12);
        \path
            [black,draw,thick]
            (PHRT) ++(16,-9)
            -- ++(1,0) -- ++(0,12) -- ++(6.25,0) -- ++(0,-5.5)
            (PHRT) ++(24,-9)
            -- ++(1,0) -- ++(0,-12);
        \path
            [black,draw,thick]
            (PHRT) ++(12,6)
            -- ++(3,0) -- ++(0,-8);
        \end{tikzpicture}
    \caption{Three branch predictors.  From left to right:  One-level predictor; Gshare; and two-level Yeh predictor.  12-bit wide registers are shift registers keeping the branch history.  An Agree predictor may branch when two or more different predictors agree.}
    \end{center}
    \label{fig:Kerberose-yeh_branch_predictor}
\end{figure}

