%Instruction-level parallelism

%[I]nstruction, [r-d]estination, [r-s]ource

%Destination = read, source = write


%pc=1000 I rd1, rs2  r1→r1a
%pc=1001 I rd2, rs3  r2→r2a
%pc=1002 I rd3, rs2  r3→r3a, !r2←r2a(1001) [r2a final dependency: 1002]
%pc=1003 I rd4, rs2  r4→r4a, !r2←r2a(1001) [r2a final dependency: 1003]
%pc=1004 I rd5, rs3  r5→r5a, !r3←r3a(1002) [r3a final dependency: 1004]
%pc=1005 I rd2, rs4  r2→r2b, !r4←r4a(1001) [r4a final dependency: 1005] [r2a retired, free after pc = 1003]
%pc=1006 I rd6, rs2  r6→r6a, !r2←r2b(1005) [r2b final dependency: 1006]
%...


% New instruction:  Set read dependency on current source-registers as aliased;
%  Update final dependency on those source registers to current pc
%
% Rename destination register; update alias at current pc to point to destination register
%
% Check read dependency on non-retired data written by prior instructions
%  If dependency, set aside
%  Else, dispatch
%
% Instruction complete:  forward data back to Load; accounting back to Dispatch
%
% Memory access:  Buffer memory writes relative to highest executed pc; signal register checkpoint
% to one instruction prior to lowest non-executed pc and writeback buffer to RAM
%
% NOTE:  Need a fast way to cross-check a set of instructions against one another in parallel.
% Possibly by a first-select MUX to parallel evaluate interdependencies.  Multi-stage is also
% possible for huge instruction-per-clock, but creates a longer pipeline with large IPC costs to
% branch misprediction.


% Horizontal?
%
%      1000 1001 1002 1003
%
% Read   r1   r1   r2   r1
% Write  r2   r3   r4   r5
%
%



\begin{figure}
    \begin{center}
        \begin{tikzpicture}[auto]
            \node (A) at (0,0) {A};
            \draw (A) ++(2,0) node (B) {B};
            \draw (B) ++(2,0) node (C) {C};

            \tikzstyle{every node}+=[black, thick, fill=white]
            % AND gate
            \node[and gate US, draw] at ($(B) + (1,-2)$) (BAnd) {};
            \node[and gate US, draw, fill=white] at ($(C) + (1,-2)$) (CAnd) {};
            \node[not gate US, draw, fill=white] at ($(A) + (0.5,-1)$) (ANot) {};
            \node[not gate US, draw, fill=white] at ($(BAnd.output) + (0.5,0)$) (BNot) {};

            \draw (A) ++(0,-3) node (ASel) {$Sel_A$};
            \draw ($(BAnd.output) + (0.15,-1)$) node (BSel) {$Sel_B$};
            \draw ($(CAnd.output) + (0.15,-1)$) node (CSel) {$Sel_C$};
            \path
                [black, draw, thick]
                (A) -- (ASel)
                (A) -- ++(0,-1) |- (ANot.input)
                (ANot.output) -- ++(0.5,0) |- (BAnd.input 2);
            \path
                [black, draw, thick]
                (B) |- (BAnd.input 1)
                (C) -- ++(0,-1) -- ++(0.5,0) |- (CAnd.input 1)
                (CAnd.output) -- ++(0.15,0) -- (CSel);
            \path
                [black, draw, thick]
                (BAnd.output) |- (BNot.input)
                (BAnd.output) ++(0.15,0) -- (BSel)
                (BNot.output) -- ++(0.15,0) |- (CAnd.input 2);

%\path
%[black, draw, thick]
%(SS) ++(2.5,2) |- (SSAnd.input 1)
%(SS) ++(1,2) -- ($(ART) + (1,-8)$)
%(SS) ++(0.5,3.75) -- ++(1,1) % Slash to show multi-data bus
%(LLVP) ++(0,-5.5) |- (SSNot.input) (SSNot.output) -- ++(-0.5,0) |- (SSAnd.input 2)
%(SSAnd.output) -- ($(ART) + (3,-8)$);

        \end{tikzpicture}
\caption{Logical find-first circuit, ripple}
\end{center}
\label{fig:ilp-logical-find-first}
\end{figure}

\begin{figure}
    \begin{center}
        \begin{tikzpicture}[->,>=latex,auto]
            %    \tikzstyle{every node}+=[inner sep=0pt]
            \node (Fetch) [startstop] {Fetch};
            \node (Instruction) [io, below of=Fetch, node distance=2cm] {Instruction};
            \node (WriteBlocked) [decision, below of=Instruction, node distance=2cm] {Depends on};

        \end{tikzpicture}
        \caption{Out-of-Order Execution Flow Chart}
    \end{center}
    \label{fig:ooe-flow-chart}
\end{figure}