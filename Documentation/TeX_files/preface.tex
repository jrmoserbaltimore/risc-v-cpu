\chapter*{Preface}
This book documents the design of the Kerberos RISC-V core.  Kerberos implements RV32I and RV64I, plus the multiply and divide extension, atomics, single- and double-precision floating point operations, 16-bit compressed instructions, Zicsr, and Zifenci.  Together, this is called RV64IMAFDCZicsr\_Zifencei, or RV32-, etc..  Kerberos also implements SMT, runahead, and out-of-order execution, along with the Supervisor and User modes.

I created Kerberos largely to understand CPU design.  This lead to reading many research papers, learning VHDL and SystemVerilog, and taking lessons from projects such as ZipCPU and the Taiga RISC-V core.  The blog for ZipCPU provided lots of information, but the developers of the Taiga CPU only provided a research paper with not much in-depth detail, along with some follow-up writings.  I see this as a sizable defect, and so decided to document what I can of these things, and hopefully learn something along the way.

Chapters in this book may be short, rather than combined simply for length.  What is topical is topical, and there's no sense in mashing a bunch of different topics together to make a chapter twelve pages long.

I divided this book into three sections:  one reviewing prior work; a sizable section documenting Taiga RISC-V by Matthews and Shannon; and a section documenting the design of Kerberos.