\chapter*{Preface}
This book documents the design of the Taiga and Kerberos RISC-V cores.  Kerberos implements RV32I and RV64I, plus the multiply and divide extension, atomics, single- and double-precision floating point operations, 16-bit compressed instructions, Zicsr, and Zifenci.  Together, this is called RV64IMAFDCZicsr\_Zifencei, or RV32-, etc..  Kerberos also implements SMT, runahead, and out-of-order execution, along with the Supervisor and User modes.

I created Kerberos largely to understand CPU design.  This lead to reading many research papers, learning VHDL and SystemVerilog, and taking lessons from projects such as ZipCPU and the Taiga RISC-V core.  The blog for ZipCPU provided lots of information, but the developers of the Taiga CPU only provided a research paper with not much in-depth detail, along with some follow-up writings.  Because Taiga is so interesting, I decided to look more deeply into it in particular and document lessons learned from its implementation.

Chapters in this book may be short, rather than combined simply for length.  What is topical is topical, and there's no sense in mashing a bunch of different topics together to make a chapter twelve pages long.

This document facilitate free and independent learning.  Free tools and low-cost development boards for high-performance FPGAs and FPGA-SoCs make digital design highly-accessible today.  Programming skills translate well to modern hardware definition languages (HDLs) like SystemVerilog, so long as the designer learns a little about digital versus software design.

While nobody is going to engineer a modern high-end processor from what they read on Wikipedia, the average person with no background in digital design can readily design a basic CPU.  A grasp of combinational logic—whereas programs execute sequentially, combinational logic executes continuously, timed by edge-sensitive flip-flops hooked up to clocks—allows competent exploration of digital design.  Digital design is also more-demanding of planning practices required in programming, including state machine and block diagrams.

I divided this book into sections reviewing prior work and a section documenting the design of Kerberos.