\chapter{Pipeline}

\section{Skid Buffer}

I built the Kerberos skid buffer based on excellent explanations at the ZipCPU
blog \footcite{ZipCPU.Pipeline}.  \prettyref{fig:pipeline-skid-buffer-fsm}
shows the state machine for the skid buffer.
\prettyref{tab:skid-buffer-handshake-state} shows information about each state;
as noted on the ZipCPU blog, \inlinecode{$In.Busy = FullBuffer$}.

\afterpage{
\begin{table}[ht]
    \caption{table}{Handshake States} % title of Table
    \centering % used for centering table
    \begin{tabular}{c c c c} % centered columns (4 columns)
        \hline\hline
        State & FullBuffer & Out.Strobe & DataOut \\ [0.5ex] % inserts table

        \hline
        Passthrough & 0 & In.Strobe & Din \\
        Buffer & 1 & In.Strobe & Din \\
        Flush & 0 & 1 & Buffer \\ [1ex]
        \hline
    \end{tabular}
    \label{tab:skid-buffer-handshake-state}
\end{table}

    % Pipeline FSM
\begin{figure}
    \begin{center}
        \begin{tikzpicture}[->,>=latex,auto]
    %    \tikzstyle{every node}+=[inner sep=0pt]

        \node[initial,accepting,state,draw=blue] (Pass) {$Pass$};
        \node[state,fill=red,text=white] (Buf) [right of=Pass, node distance=8cm] {$Buffer$};
        \node[accepting,state,fill=green] (Flush) [below left of=Buf, node distance=6cm] {$Flush$};

        \path (Pass)  edge [loop above] node {!In.Strobe $||$ !Out.Busy} (Pass)
                      edge node {In.Strobe \&\& Out.Busy} (Buf)
              (Buf)   edge [loop above] node {Out.Busy} (Buf)
                      edge [bend left] node {!Out.Busy} (Flush) % Bend right
              (Flush) edge [bend left] node {In.Strobe \&\& Out.Busy} (Buf)
                      edge [bend left] node {!In.Strobe $||$ !Out.Busy} (Pass);
        \end{tikzpicture}
        \caption{Handshake State Machine}
    \end{center}
    \label{fig:pipeline-skid-buffer-fsm}
\end{figure}
}

Pipeline stages signal $Busy$ whenever they are in the $Run$ or $Wait$ states.
A circuit is in the $Run$ state while it is processing, and in the $Wait$ state
when it has data sitting on its output but the next stage remains $Busy$.  When
the circuit is neither strobing to the next circuit nor processing, it becomes
$Idle$, and is not $Busy$ even if the next stage becomes $Busy$. This produces
four states:  Idle, Run, Wait, and Finish, shown in
\prettyref{tab:pipeline-exec-circuit-state}.

The circuit passes its data on in the Finish state, and is idle on the next
iteration; however, the busy signal is \inlinecode{Processing || (DataReady
\&\& PipeIn.Busy)}, and so signals \inlinecode{!Busy} on the cycle on which it
delivers data.  The \inlinecode{SkidBuffer} itself must accept the data and,
besides, is \inlinecode{!Busy} when its buffer is empty and will not magically
become busy.

\afterpage{
\begin{table}[ht]
    \caption{Execution Circuit Handshake States} % title of Table
    \centering % used for centering table
    \begin{tabular}{c c c c c} % centered columns (4 columns)
        \hline\hline
        State & Processing & DataReady & PipeOut.Busy & PipeIn.Busy \\ [0.5ex] % inserts table

        \hline
        Idle & 0 & 0 & X & 0 \\
        Run & 1 & 0 & X & 1 \\
        Wait & 0 & 1 & 1 & 1 \\
        Finished & 0 & 1 & 0 & 0 \\ [1ex]
        \hline
    \end{tabular}
    \label{tab:pipeline-exec-circuit-state}
\end{table}
\begin{figure}
    \begin{center}
        \begin{tikzpicture}[->,>=latex,auto]
            %    \tikzstyle{every node}+=[inner sep=0pt]

            \node[initial,accepting,state,draw=blue,text=black] (Idle) {Idle};
            \node[state,fill=red,text=white] (Run) [right of=Idle, node distance=8cm] {Run};
            \node[state,fill=red,text=white] (Wait) [below of=Run, node distance=5cm] {Wait};
            \node[accepting,state,fill=green] (Finish) [below of=Idle, node distance=5cm] {$Flush$};

            \path (Idle)   edge [loop above] node {!PipeIn.Strobe} (Pass)
            edge node {PipeIn.Strobe} (Run)
            (Run)    edge [loop right] node {!DataReady} (Run)
            edge node {DataReady} (Wait)
            (Wait)   edge [loop right] node {PipeOut.Busy} (Wait)
            edge node {!PipeOut.Busy} (Finish)
            (Finish) edge node {PipeIn.Strobe} (Run)
            edge node {!PipeIn.Strobe} (Idle);
        \end{tikzpicture}
        \caption{Execution Circuit Handshake State Machine}
    \end{center}
\end{figure}
}

\section{Fetch}

The \inlinecode{Fetch} stage obtains new instructions and passes them through
the pipeline.  It can pass forward large amounts of data—64-byte cache
lines—containing several instructions, along with the value of \inlinecode{pc}
after a branch.  \inlinecode{Fetch} always retrieves and passes along a chunk
of instruction stream aligned to its chunk size.


\section{Pre-Decode}

\inlinecode{Pre-Decode} prepares the instruction stream for
\inlinecode{Decode}.  It receives instruction stream from \inlinecode{Fetch}
one chunk at a time, and tracks \inlinecode{pc} to identify the current
address.

\inlinecode{Pre-Decode} converts RISC-V Compressed instructions (RVC) to normal
RISC-V instructions.  RVC includes all opcodes with the two least-significant
bits not equal to \inlinecode{11}, so \inlinecode{Decode} can ignore these two
bits.  This allows cheap handling of \inlinecode{C.JAL} and
\inlinecode{C.JALR}:  RVC passes the lower two bits of the opcode forward
as-is, and the \inlinecode{Branch} circuit adds 2 rather than 4 to
\inlinecode{pc} when these bits are not \inlinecode{11}.

\inlinecode{Pre-Decode} separates each quadrant of RVC into an independent
combinational circuit and selects the output via a mux.

Whether expanded from RVC or passed forward verbatim, \inlinecode{Pre-Decode}
assembles a single 32-bit instruction with its context information for
\inlinecode{Decode}.

\section{Decode}

\inlinecode{Decode} converts instructions to an internal bundle of signals
indicating the operation, width, signed or unsigned nature, and instruction
layout type.  Like \inlinecode{Pre-Decode}, groups of logically-similar
instructions execute as independent combinational circuits and raise a signal
to indicate an identified instruction.

Often a single opcode maps to only one or two formats, and the
\inlinecode{funct3} and \inlinecode{funct7} fields indicate data width, sign,
or modes such as shift-right or subtract, so these circuits are quite compact
in gate logic.  A 5-LUT can decode an opcode, and often a 5-LUT or 6-LUT can
look up the remaining information—often one bit in the opcode, three from
\inlinecode{funct3}, and one from \inlinecode{funct7}.  This minimizes the
logic used on FPGAs.

\section{Dispatch}

\inlinecode{Dispatch} determines if instructions have dependencies on pending
instructions, directs register renaming, and distributes instructions across
multiple execution units if present\footnote{Branches and jumps are always
memory and register barriers; branches cause a stall unless branch prediction
and speculative execution are implemented.  These require additional
checkpointing.  Branches incur a five-cycle stall without prediction, and tight
loops suffer significant loss in IPC.}.

\section{Load}

\inlinecode{Load} loads data from registers and extracts immediate values from
most instructions.  \inlinecode{Execute} derives immediate values from
two-register instructions, notably \inlinecode{Branch} instructions.

\section{Execute}

Small numbers above and below the execute operation indicate the number of
cycles and the latency per execution unit.  For example, if the execution stage
contains two dividers, the latency can be one cycle rather than however long it
takes for the divider to complete.

The \inlinecode{Branch} and \inlinecode{Load/Store} operations use the ALU for
simultaneous addition and comparison.  \inlinecode{Branch} caches the
\inlinecode{pc} and computed address of the prior interpreted branch
instruction, using the cached target address if the branch condition and
\\inlinecode{pc} match\footnote{Speculative adders occasionally run for two
cycles, in which case so will \inlinecode{Load/Store}.  Tight loops use
backwards branches and suffer a significant performance loss if the addition
requires an extra cycle.  The ALU supplies two comparators to allow
simultaneous comparison of \inlinecode{pc} and the branch condition.}.

\inlinecode{Branch} jumps to the the cached target address if the branch
condition and prior branch instruction match, avoiding the adder for all but
the first iteration of a tight loop.  \inlinecode{Branch} flushes the entire
pipeline when taking a non-predicted branch.

\section{Memory Access and Cache}

\inlinecode{Memory Access} also forwards register loads back through the
pipeline.  Both \inlinecode{MemoryAccess} and \inlinecode{Fetch} connect
directly to VC-DSR L1 cache.

At higher clock rates, L1 cache hits may respond in as many as 5 cycles.  A
single cache line holds 64 bits, enough for 16 instructions or 32 compressed
instructions.  Vector instructions can be highly-efficient, using single
full-width execution units to perform operations on multiple smaller-width
data, and fail to saturate the pipeline if it's designed for more than 12
simultaneous operations.  The pipeline can fuse 32-bit instructions in the same
manner, when independent, although this is not implemented.

Instruction fetch thus requires fast trace caching and, more importantly,
highly-efficient branch prediction.
