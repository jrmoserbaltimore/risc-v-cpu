% vim: ts=4 sw=4 et
\chapter{Speculative Execution}

This chapter covers speculative execution.

\section{Security}

Speculative execution causes side effects:

\begin{itemize}

    \item Cache is loaded or invalidated

    \item Branch prediction tables are updated

\end{itemize}

These side effects are usually harmless.  Runahead, for example, continues
executing during a D\$ miss to speculatively load the cache.  Because the
registers and memory buffer contain no privileged information, runahead can't
access any information not available to the thread normally, and so cannot leak
information.

These guarantees fail when processors contain severe design flaws.

\subsection{Impermissible Cache Loads}

Meltdown leverages impermissible loads by using the data as an offset into an
uncached memory area and searching for a statistically-fast read indicating
cached data.  Meltdown relies on speculative execution issuing these reads
after accessing a supervisor page.

Kerberos's VC-DSR stores the ASID and permissions for each cache line's leading
virtual page and for each virtual page synonym to the leading virtual page.
D\$ does not provide instructions to the pipeline, so these permissions include
only R, W, and U.  Any load or store operation must supply the ASID and whether
it's reading or writing; adding the U bit allows Kerberos to short-circuit a
data load in a number of ways:

\begin{itemize}

    \item A D\$ hit finds a permission mismatch, and D\$ returns a protection
        fault

    \item A D\$ miss finds an entry in the ART, TLB, or ASDT with a permission
        mismatch, and D\$ returns a protection fault

    \item A D\$ TLB miss causes a page table walk and finds a permission
        mismatch, and D\$ returns a protection fault

\end{itemize}

\subsection{L1 Cache Address Translation}

Modern caches are virtually-indexed, physically-tagged (VIPT), and need to look
up physical addresses in certain situations.  This mechanism in particular
caused Foreshadow and Foreshadow-NG, where a program reads arbitrary data from
L1 cache before the tag is resolved.

Kerberos uses a VIVT cache and isn't subject to this.

\begin{itemize}

    \item Cache line tags include ASID and, for hypervisor extensions, VMID,
        associating them with an address space context to avoid homonyms.

    \item Synonyms aren't even vaguely related to the index and tag they
        reference, so speculation ahead of an ART lookup would be bizarre.

    \item Repeated access to the same page uses a LLVP register to avoid the
        ART and avoid the SS false positive condition; a larger SS dramatically
        reduces the false positive rate, while speculating in the manner by
        which Foreshadow occurs would be enormously more expensive.

    \item Global pages bypass ART in that ASIDs aren't factored into their
        synonym check.

    \item Physical address information must be looked to determine if an L1
        miss is real, and is available for L2 lookups, at which point no
        further misreads are possible.

\end{itemize}

Because of all this, VC-DSR cannot function in a manner by which speculative
cache fetches provide any use.

\subsection{Store Bypass}

A store buffer contains only data from the current context and cannot contain
impermissible data.  Specter Speculative Store Bypass (SSB) leverages
misaligned loads and stores to pull
rogue system register read

\subsection{Branch Speculation}

{\bf FIXME:  very rough, mostly tacked down as scribbled notes}

Spectre leverages branch misprediction and speculative execution.

Fully-associative caches do not correlate the specific cache line with a
specific address due to not being indexed.

Number of evicted cache lines is abusable:

\begin{itemize}

    \item Mispredict branch

    \item Load register

    \item begin speculating using register

    \item Start looping through data using register counter

    \item Drop results, take correct branch

    \item Loop through a previously-cached array in reverse and detect last
        cached address

\end{itemize}

Speculation must wait for D\$ to accurately load the register; and further
reads and computations require cached data.  If the data is in cache, this
attack doesn't work; if the data is not in cache, further execution requires
loading the data for accurate results.

During a D\$ miss, Kerberos enters runahead mode, tracking valid and invalid
register and store buffer data and using this to predict valid and invalid
state.  Invalid state may later be proven valid.  Crucially, a predicted branch
is speculative until computed and confirmed.

One of two things will happen during runahead:  the branch will be computed and
the runahead path will be invalidated, or the branch will be computed and the
runahead path will be as valid as it can be in non-branching code.  If the
branch is invalidated, all loads are not valid; if it is validated, {\em some}
loads are valid.

Runahead and speculative execution operate on this principle:  runahead skips
loads and marks registers as not valid when the loads are not known valid,
while speculation notates the results of operations and if they're on
known-valid data.  In Kerberos, runahead doesn't queue up loads in D\$; rather
it queues {\em speculative} loads and flags a validity condition in a bitfield.
When the validity condition is proven true, D\$ performs the load.

For example:  runahead passing a predicted branch will mark all queued loads as
predicated on a validity condition.  When execution proves the branch is
taken—such as after a register load completes and the branch requiring it is
immediately computed to verify the prediction—runahead signals D\$ that the
validity condition is proven.  D\$ will then evict cache lines and load data.

This approach must meet the following constraints:

\begin{itemize}

    \item D\$ has a queue of known-valid items that {\em will} be loaded in the
        execution path, so loading them is harmless.

    \item D\$ does not load or invalidate cache lines until it has a {\em
        valid} address to load.  This means branches leading to the read are
        100\% determined to be taken or not taken, registers used to calculate
        addresses are 100\% determined to contain the expected contents, and so
        forth.

    \item D\$ may issue speculative TLB and ASDT lookups for addresses not in a
        currently-cached LVP, but won't issue these for addresses in a
        currently-cached LVP.  These speculative lookups cannot be used until
        {\em one} address in the referenced virtual page is made valid in L1
        cache, as after the item is valid there is no remaining timing attack
        against the TLB and ASDT.

    \item When a constraint is proven incorrect, D\$ invalidates all related
        queued speculative loads.
\end{itemize}

When a constraint is proven incorrect, no further action is needed except the invalidation of queued speculative loads.  This can be done by simply never validating them {\em if} runahead validity state is managed, which is necessary for this approach.

The processor must indicate the correct computations for upcoming branches to runahead even when runahead isn't running; it must forward valid loads and their source addresses for runahead to compare with its predictions; and so forth.  For a lazy approach, access to a speculative D\$ load must trigger runahead, which then {\em confirms} its validity conditions in a 100\% accurate manner, updates the D\$ speculative queue, and continue runahead using the more-valid state.

This approach ensures CPU behavior can leak state of which the processor was certain, but no information about state of which the processor was uncertain.  Two factors guarantee this:

\begin{itemize}

    \item Prefetch operations such as runahead must not evict cache data not
        yet used by a pending operation to prefetch a cache line used in a
        later operation—this would cause cache misses and decrease
        performance—so software cannot identify future state by looking for
        unexpected cache misses

    \item No cached information shall be used until it is confirmed
        non-speculative, so software can only determine future state from state
        it already has carried out to its correct conclusion, which is what
        software does.

\end{itemize}

This has odd implications, e.g. an incorrectly-predicted loop attempting to
read through an array to extract timing from future execution will remain
marked as speculative and then determined mispredicted, and the timing effects
of cache speculation won't arise.  As soon as the loop is entered, the
misprediction leads to a cache miss which triggers runahead and rapidly loads
cache with correct data.  If the loop is small, it may validate future data and
cause further prefetching, but by nature this can only occur with
confirmed-valid state.

This approach retains the benefits of speculation and prevents the information
leaks brought on by branch misprediction.

\section{Branch Prediction}

Three-block fetch with expensive, complicated, and power-hungry L\$ can fill in
for this when using a good branch predictor.  Modern processors use a trace
cache to bridge between branches, following the same instructions across cache
lines through a much simpler and faster buffer containing a program-order copy
of the reassembled instruction stream; this performs little if any better than
perfect branch prediction when the branch target hits the fetch stage just in
time, but branch predictors aren't perfect.

Kerberos uses a lightweight hybrid predictor consisting of an agree predictor,
a prioritized runahead buffer, and a static predictor.  It also provides a loop
predictor, which identifies when a backwards branch target address is greater
than or equal to the next branch instruction; and an indirect branch predictor
which calculates the target of \lstinline!jalr!.  Unconditional branches are
not tracked in any predictor.

Kerberos considers indirect branches computable when there are several
instructions between \lstinline!jalr rd, n(rx)! and the most recent instruction
setting \lstinline!rx!.  The branch predictor computes this in parallel and
stores it into a one-entry indirect branch register.  When the indirect branch
register

--using a 16-bit parallel prefix adder tied to a carry-select high-speed increment chain—the selectable alternative is the input value and there is a mux for every 16 bits to finish roughly in time

The agree predictor XORs the output of a global shared predictor and a local
predictor to determine strong (0) or weak (1) prediction, as below.

\begin{table}[htp]
    \centering
\begin{tabular}{|c|c|c|}
    \hline
    \multicolumn{2}{|c|}{Predictors} & Prediction \\
    \hline
    Gshare & Local & Agree \\
    \hline
    0 & 0 & Not Taken \\
    \hline
    0 & 1 & Weak Taken \\
    \hline
    1 & 0 & Weak Taken \\
    \hline
    1 & 1 & Taken \\
    \hline
\end{tabular}
\caption{Agree predictor}
\label{tab:branch-agree-predictor}
\end{table}

The static predictor uses the direction to predict the branch.  When the agree
predictor produces a weak-taken result, the static predictor breaks the tie by
using a direction heuristic.

\begin{table}[htp]
    \centering
\begin{tabular}{|c|c|c|}
    \hline
    Agree & Direction & Prediction \\
    \hline
    Taken & Forward & Taken \\
    \hline
    Taken & Backward & Taken \\
    \hline
    Weak Taken & Forward & Not Taken \\
    \hline
    Weak Taken & Backward & Taken \\
    \hline
    Not Taken & Forward & Not Taken \\
    \hline
    Not Taken & Backward & Not Taken \\
    \hline
\end{tabular}
\caption{Predictions accounting for branch direction}
\label{tab:branch-static-hybrid}
\end{table}

Finally, the runahead predictor provides information.  Branches followed using
values from only valid registers are considered definitely taken or not taken;
branches followed using not-valid registers are considered weakly taken or not
taken.

\begin{table}[htp]
    \centering
    \begin{tabular}{|c|c|c|}
        \hline
        Agree & Runahead & Prediction \\
        \hline
        Strong Prediction & Strong Prediction & Runahead \\
        \hline
        Weak Taken & Strong Prediction & Runahead \\
        \hline
        Weak Taken & Weak Taken & Taken \\
        \hline
        Weak Taken & Weak Not Taken & Static \\
        \hline
        Strong Prediction & Weak Prediction & Agree \\
        \hline
        Strong Prediction & No Prediction & Agree \\
        \hline
        Weak Taken & No Prediction & Static \\
        \hline
    \end{tabular}
\caption{Predictions including runahead}
\end{table}

\subsection{Global Shared Predictor}

The global shared predictor (Gshare) uses an 8- to 12-bit Branch History Table
(BHT) providing 256 to 4096 entries in 64 to 1024 bytes of storage.

When a branch is encountered, the lower bits of \lstinline!pc<<2! are XOR'd
with a branch history register (BHR) containing the branch history to create a
BHT index.  The predictor updates the two-bit BHT entry at this index, shifts
the register is shifted one left and the LSB set to the branch outcome, and the
two-bit Smith entry in the BHT update

\subsection{Two-Level Adaptive}

The Yeh algorithm is an option in place of Gshare, and can be much more
accurate with large pattern tables.  The predictor for Yeh uses a pattern
history table (PHT) and a Per-Address History Register Table (PHRT).


\begin{figure}
% Kerberos overview, similar to Taiga diagram
    \begin{center}
        \begin{tikzpicture}[scale=0.2]
            \tikzstyle{every node}+=[inner sep=0pt]

            % Global BHR
            \path
                 % Node and label
                 [black,draw,thick]
                 % Origin
                 (0,0) node [black] (Global BHR) {}
                 ++(6,2) node {\small{Global BHR}};
            % Shift register
            \foreach \x in {0,...,11}
                \path [black, draw, thick]
                 (Global BHR) ++(\x,0) -- ++(1,0) -- ++(0,-1) -- ++(-1,0) -- ++(0,1) -- cycle;
             % Grey boxes
             \foreach \x in {1,3,4,7,8,11}
                \path [black, draw, thick, fill=gray]
             (Global BHR) ++(\x,0) -- ++(1,0) -- ++(0,-1) -- ++(-1,0) -- ++(0,1) -- cycle;

            \path
                [black,draw,thick]
                (Global BHR) ++(18,-4) node (PHRT) {}
              ++(6,2) node {\small{PHRT}};
            \path
                [black,draw,thick]
                (PHRT) ++(18,0) node (PT) {}
              ++(2,2) node {\small{PT}};

            \path
                [black,draw,thick]
                (Global BHR) ++(-3,-4) node (Local BHT) {}
                ++(4,-6) node [rotate=90]{\small{Local BHT}};

            % Local BHT
            \path
                [black,draw,thick]
                (Global BHR) ++(10,-4.5) node (Global BHT) {}
              ++(4,-6) node [rotate=90]{\small{Global BHT}};
            \foreach \y in {0,...,11}
                \foreach \x in {0,1,13,14,21,22,...,33,40,41}
                  \path [black, draw, thick]
                  (Local BHT)
                ++($(\x,0) + -1.5*(0,\y)$)
                  -- ++(1,0) -- ++(0,-1) -- ++(-1,0) -- ++(0,1) -- cycle;

%            \foreach \y in {0,...,11}
%                \foreach \x in {14,...,15}
%                \path [black, draw, thick]
%                (Local BHT)
%              ++($(\x,0) + -1.5*(0,\y)$)
%                -- ++(1,0) -- ++(0,-1) -- ++(-1,0) -- ++(0,1) -- cycle;

            %%%%%% All this to make it look random
            \foreach \y in {0,...,11}
                \foreach \x in {0,1,13,14,21,22,...,33,40,41}
                \ifthenelse{
                   \intcalcMod{
                     \intcalcAdd{
                      \x}{
                      \intcalcMul{\y}{7}
                     }
                   }{3}
                > 0 \AND
                \intcalcMod{
                    \intcalcAdd{
                        \y}{
                        \intcalcMul{\x}{11}
                    }
                }{7} > 1
            }{
                  \path [black, draw, thick, fill=gray]
                    (Local BHT)
                ++($(\x,0) + -1.5*(0,\y)$)
                -- ++(1,0) -- ++(0,-1) -- ++(-1,0) -- ++(0,1) -- cycle}{};
            %%%%% Never try to read that ugly code

        \path
            [black,draw,thick]
            (Local BHT) ++(-1,0)
            -- ++(0,-18) -- ++(-2,4) -- ++(0,10) -- ++(2,4) -- cycle
            (Local BHT)
          ++(-8, 6) node {Address};
        \path
            [black,draw,thick]
            (Global BHR) ++(9,-4)
            -- ++(0,-18) -- ++(-2,4) -- ++(0,10) -- ++(2,4) -- cycle;
        \path
            [black,draw,thick]
            (PHRT) ++(14,0)
            -- ++(0,-18) -- ++(2,4) -- ++(0,10) -- ++(-2,4) -- cycle
            (PHRT)
          ++(8,6) node {Address}
            (PT) ++(4,0)
            -- ++(0,-18) -- ++(2,4) -- ++(0,10) -- ++(-2,4) -- cycle;
        %%%% MUX traces
        \path
            [black,draw,thick]
            (Local BHT) ++(-3,-9)
            -- ++(-1,0) -- ++(0,-12)
            (Global BHR) ++(7,-13)
            -- ++(-1,0) -- ++(0,-12)
            (PHRT) ++(16,-9);
        \path
            [black,draw,thick]
            (Local BHT) ++(-4,6)
            -- ++(2,0) -- ++(0,-8);
        \path[black,draw,thick]
            (Global BHR) ++(8,-1)
            -- ++(0,-5);

        \path
            [black,draw,thick]
            (PHRT) ++(16,-9)
            -- ++(1,0) -- ++(0,12) -- ++(6.25,0) -- ++(0,-5.5)
            (PHRT) ++(24,-9)
            -- ++(1,0) -- ++(0,-12);
        \path
            [black,draw,thick]
            (PHRT) ++(16,-9)
            -- ++(1,0) -- ++(0,12) -- ++(6.25,0) -- ++(0,-5.5)
            (PHRT) ++(24,-9)
            -- ++(1,0) -- ++(0,-12);
        \path
            [black,draw,thick]
            (PHRT) ++(12,6)
            -- ++(3,0) -- ++(0,-8);
        \end{tikzpicture}
    \caption{Three branch predictors.  From left to right:  One-level predictor; Gshare; and two-level Yeh predictor.  12-bit wide registers are shift registers keeping the branch history.  An Agree predictor may branch when two or more different predictors agree.}
    \end{center}
    \label{fig:Kerberose-yeh_branch_predictor}
\end{figure}



Whereas Gshare hashes the branch address with the branch history to look up
history in a global BHT, Yeh looks up the branch address in a table and tracks
its own branch history there—a local branch predictor.
