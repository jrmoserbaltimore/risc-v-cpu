\chapter{Cache}

A highly-parallel, superscalar CPU places high demand on memory bandwidth and
latency.  Instruction fetch must supply the pipeline with operations to
execute; load instructions must wait for memory access; branches clear out the
pipeline, equivalent to stalling once the branch is fetched until the branch is
taken.

Cache isn't as simple as putting memory close to the pipeline.  A 64-byte cache
line holds 16 full-width instructions or 32 RVC instructions.  Even with
single-cycle execution, this gives plenty of L\$ cache access time—the Pentium
Pro had a 3-cycle L1 hit latency, and modern CPUs run even as much as 5-cycle
L\$ latency.

Superscalar processors can execute more than one instruction per clock (IPC),
demanding more from cache.  Successfully executing 32 instructions each clock
can stall the CPU for 67\% to 80\% of its execution.  While vector instructions
can saturate a superscalar pipeline, this only happens in tight and
well-predicted loops with effective cache prefetching.

More advanced methods are necessary.

\section{Cache}

Kerberos combines several cache technologies.

\subsection{VC-DSR Virtually Cache}

Kerberos uses Virtual Cache with Dynamic Synonym Remapping (VC-DSR), a
virtually-indexed, virtually-tagged (VIVT) VC-DSR L1 cache.  L2 and beyond are
physically-indexed, physically-tagged (PIPT).  I\$ and D\$ use separate Active
Synonym Detection Tables (ASDTs).  This VC-DSR implementation uses last leading
virtual page (LLVP) caching and ignores ASIDs for global pages to avoid ART
look-ups via detecting them as homonyms.

On miss, L1 cache accesses an Active Synonym Detection (ASD) arbiter.  If the
data is in L1 cache and the address remapping table (ART) does not include the
virtual page as a synonym to the leading virtual page (LVP), the Translation
Lookaside Buffer (TLB) and Active Synonym Detection Table (ASDT) provide this
information and replay the cache request.  The TLB may perform a page table
walk through the MMU during this process and check the resulting physical page
against the ASDT.

When the ASDT does not contain a match, the ASD arbiter consults further memory
systems, such as L2 cache or main memory.  Multi-core systems may use an L3
cache shared between cores; FPGA implementation may implement L2 or L3 on
non-paging external RAM, such as 16Mib (2Mio) of 10ns static RAM arranged with
64-bit word size (around \$8).

\subsection{Way Cache}

A 4KiB cache contains 64 entries, and dividing into an 8-way set-associative
cache places 8 entries into each index.  Kerberos uses a 16-index
set-associative cache above 16KiB, giving 16-way set associative for 16KiB,
64-way for 64KiB.

Kerberos uses a way cache\footcite{Nicolaescu2005} for L1 cache.  The table
below shows various cache sizes, associativity, and the size of a way cache
sized to index all cache lines.

\begin{table}[htp]
    \centering
    \begin{tabular}{|c|c|c|c|c|}
        \hline
        Cache size & Cache Lines & Ways & Way Cache & Ratio \\
        \hline
        4KiB & 64 & 8 & 24 bytes & 0.59\% (\nicefrac{1}{171}) \\
        \hline
        4KiB & 64 & Fully-associative & 48 bytes & 1.2\% (\nicefrac{1}{85}) \\
        \hline
        64KiB & 1024 & 64 & 768 bytes & 1.2\% (\nicefrac{1}{85}) \\
        \hline
        64KiB & 1024 & Fully-associative & 1.125KiB & 1.8\% (\nicefrac{1}{57}) \\
        \hline
        1024KiB & 16384 & 64 & 12KiB & 0.12\% (\nicefrac{1}{85}) \\
        \hline
        1024KiB & 16384 & 1024 & 1.125KiB & 0.11\% (\nicefrac{1}{910}) \\
        \hline
        1024KiB & 16384 & Fully-associative & 26KiB & 2.6\% (\nicefrac{1}{39}) \\
        \hline
    \end{tabular}
    \caption{Agree predictor}
    \label{tab:way-cache-size}
\end{table}

For the same associativity, the way cache remains at the same ratio.  Larger
associativity increases the number of entries in the way cache and the number
of bits per entry.
