\chapter{Parallelism}

\section{Instruction Level Parallelism}

Kerberos uses a parallel instruction pipeline with multiple execution units.
The Fetch stage obtains several instructions from I\$ and fans them out to
multiple decoder pipelines, which then fan them in to the Dispatch stage as
shown in \prettyref{fig:ilp-pipeline-overview}.

\begin{figure}
    \begin{center}
        \begin{tikzpicture}[->,>=latex,auto]

        \node[initial,accepting,state,draw=blue] (Fetch) {Fetch};

        \tikzstyle{every node}+=[draw=orange]
        \node[state] (Pre-Decode1) [below of=Fetch, node distance=3cm] {Pre-Decode};
        \node[state] (Pre-Decode2) [left of=Pre-Decode1, node distance=4cm] {Pre-Decode};
        \node[state] (Pre-Decode3) [right of=Pre-Decode1, node distance=4cm] {Pre-Decode};

        \tikzstyle{every node}+=[draw=black,fill=yellow]
        \foreach \x in {1,2,3}
        \node[state] (Decode\x) [below of=Pre-Decode\x, node distance=3cm] {Decode};

        \tikzstyle{every node}+=[fill=white]

        \node[state,fill=green] (Dispatch) [below of=Decode1, node distance=3cm] {Dispatch};

        \node[state,draw=violet] (Load) [below of=Dispatch, node distance=3cm] {Load};

        \tikzstyle{every node}+=[draw=violet,fill=cyan]

        \node[state] (Execute1) [below of=Load, node distance=3cm] {Execute};
        \node[state] (Execute2) [left of=Execute1, node distance=4cm] {Execute};
        \node[state] (Execute3) [right of=Execute1, node distance=4cm] {Execute};

        \tikzstyle{every node}+=[fill=white]
        \node[state] (RMA) [below of=Execute1, node distance=3cm, text width=2cm, align=center] {Retire or Memory Access};

        \foreach \x in {1,2,3}
        \path
            (Fetch) edge (Pre-Decode\x)
            (Pre-Decode\x) edge (Decode\x)
            (Decode\x) edge (Dispatch)
            (Load) edge (Execute\x)
            (Execute\x) edge (RMA);

        \path (Dispatch) edge (Load);

        \end{tikzpicture}
        \caption{Instruction-Level Parallelism Overview}
    \end{center}
    \label{fig:ilp-pipeline-overview}
\end{figure}

Both $Dispatch$ and $Load$ track instruction dependencies; only $Load$ tracks
register contents.  $Dispatch$ determines dependencies and indicates actions
such as register renaming.

When $Dispatch$ executes instructions out of order, it indicates a register ID
from which to read each value and one to which to write any output.  $Dispatch$
delays instructions dependent on a prior write; instructions reading a register
altered by a future write use the prior register ID of that register.  This
allows independent instructions to execute out of order even when using the
same temporary registers.

$Dispatch$ and $Load$ keep a table of registers, their current aliases, and a
\inlinecode{pc} tag.  $Dispatch$ uses the current table to order instructions
and perform renaming, using a free register FIFO.  $Load$ keeps track of
register contents.  Each retiring instruction forwards the relevant information
back so both can keep track of the current register file checkpoint in terms of
\inlinecode{pc}.  The $Retire or Memory Access$ stage also tracks this
information and buffers memory writes to stay in sync with the settled register
checkpoint.

%Instruction-level parallelism

%[I]nstruction, [r-d]estination, [r-s]ource

%Destination = read, source = write


%pc=1000 I rd1, rs2  r1→r1a
%pc=1001 I rd2, rs3  r2→r2a
%pc=1002 I rd3, rs2  r3→r3a, !r2←r2a(1001) [r2a final dependency: 1002]
%pc=1003 I rd4, rs2  r4→r4a, !r2←r2a(1001) [r2a final dependency: 1003]
%pc=1004 I rd5, rs3  r5→r5a, !r3←r3a(1002) [r3a final dependency: 1004]
%pc=1005 I rd2, rs4  r2→r2b, !r4←r4a(1001) [r4a final dependency: 1005] [r2a retired, free after pc = 1003]
%pc=1006 I rd6, rs2  r6→r6a, !r2←r2b(1005) [r2b final dependency: 1006]
%...


% New instruction:  Set read dependency on current source-registers as aliased;
%  Update final dependency on those source registers to current pc
%
% Rename destination register; update alias at current pc to point to destination register
%
% Check read dependency on non-retired data written by prior instructions
%  If dependency, set aside
%  Else, dispatch
%
% Instruction complete:  forward data back to Load; accounting back to Dispatch
%
% Memory access:  Buffer memory writes relative to highest executed pc; signal register checkpoint
% to one instruction prior to lowest non-executed pc and writeback buffer to RAM
%
% NOTE:  Need a fast way to cross-check a set of instructions against one another in parallel.
% Possibly by a first-select MUX to parallel evaluate interdependencies.  Multi-stage is also
% possible for huge instruction-per-clock, but creates a longer pipeline with large IPC costs to
% branch misprediction.


% Horizontal?
%
%      1000 1001 1002 1003
%
% Read   r1   r1   r2   r1
% Write  r2   r3   r4   r5
%
%



\begin{figure}
    \begin{center}
        \begin{tikzpicture}[auto]
            \node (A) at (0,0) {A};
            \draw (A) ++(2,0) node (B) {B};
            \draw (B) ++(2,0) node (C) {C};

            \tikzstyle{every node}+=[black, thick, fill=white]
            % AND gate
            \node[and gate US, draw] at ($(B) + (1,-2)$) (BAnd) {};
            \node[and gate US, draw, fill=white] at ($(C) + (1,-2)$) (CAnd) {};
            \node[not gate US, draw, fill=white] at ($(A) + (0.5,-1)$) (ANot) {};
            \node[not gate US, draw, fill=white] at ($(BAnd.output) + (0.5,0)$) (BNot) {};

            \draw (A) ++(0,-3) node (ASel) {$Sel_A$};
            \draw ($(BAnd.output) + (0.15,-1)$) node (BSel) {$Sel_B$};
            \draw ($(CAnd.output) + (0.15,-1)$) node (CSel) {$Sel_C$};
            \path
                [black, draw, thick]
                (A) -- (ASel)
                (A) -- ++(0,-1) |- (ANot.input)
                (ANot.output) -- ++(0.5,0) |- (BAnd.input 2);
            \path
                [black, draw, thick]
                (B) |- (BAnd.input 1)
                (C) -- ++(0,-1) -- ++(0.5,0) |- (CAnd.input 1)
                (CAnd.output) -- ++(0.15,0) -- (CSel);
            \path
                [black, draw, thick]
                (BAnd.output) |- (BNot.input)
                (BAnd.output) ++(0.15,0) -- (BSel)
                (BNot.output) -- ++(0.15,0) |- (CAnd.input 2);

%\path
%[black, draw, thick]
%(SS) ++(2.5,2) |- (SSAnd.input 1)
%(SS) ++(1,2) -- ($(ART) + (1,-8)$)
%(SS) ++(0.5,3.75) -- ++(1,1) % Slash to show multi-data bus
%(LLVP) ++(0,-5.5) |- (SSNot.input) (SSNot.output) -- ++(-0.5,0) |- (SSAnd.input 2)
%(SSAnd.output) -- ($(ART) + (3,-8)$);

        \end{tikzpicture}
\caption{Logical find-first circuit, ripple}
\end{center}
\label{fig:ilp-logical-find-first}
\end{figure}

\begin{figure}
    \begin{center}
        \begin{tikzpicture}[->,>=latex,auto]
            %    \tikzstyle{every node}+=[inner sep=0pt]
            \node (Fetch) [startstop] {Fetch};
            \node (Instruction) [io, below of=Fetch, node distance=2cm] {Instruction};
            \node (WriteBlocked) [decision, below of=Instruction, node distance=2cm] {Depends on};

        \end{tikzpicture}
        \caption{Out-of-Order Execution Flow Chart}
    \end{center}
    \label{fig:ooe-flow-chart}
\end{figure}

\section{Lightweight Vector Operations}

Kerberos provides lightweight vector operations using a custom user CSR.

\begin{figure*}[h!]
    {\footnotesize
        \begin{center}
            \begin{tabular}{U@{}cccc}
                \instbitrange{31}{7} &
                \instbit{6} &
                \instbit{5}{4} &
                \instbitrange{3}{0} \\
                \hline
                \multicolumn{1}{|c|}{\warl} &
                \multicolumn{1}{c|}{E} &
                \multicolumn{1}{c|}{M[1:0]} &
                \multicolumn{1}{c|}{Size[3:0] (\wlrl)} \\
                \hline
                27 & 1 & 2 & 4 \\
            \end{tabular}
        \end{center}
    }
    \vspace{-0.1in}
    \caption{Lightweight Vector Register ({\tt zvinxd}).}
    \label{zvinxreg}
\end{figure*}

Setting ``Size'' to 0 disables vector operations; other values enable vector
operations.

``E'' controls RVE mode and is always supported; ``E'' is always set in RVE
mode and cannot be unset.  When ``E'' is not set, the vector registers are x31
and x19; when ``E'' is set, the vector registers are x15 and x11.  These are
called the high and low vector registers, respectively, and named $vh$ and
$vl$.

``Size'' indicates the number of registers used as vector registers.  Size may
not be greater than 12 when When ``E'' is unset, 8 when ``M'' is non-zero, and
4 when ``E'' is set.  The high and low register sets extend from the high and
low registers down, e.g. if E is unset and Size is 3, the vector registers are
[x31,x30,x29] and [x19,x18,x17],  corresponding respectively.

These relationships also apply to the floating point registers.

When ``M'' is non-zero, x23 acts as a vector mask:  operations only apply to
vector elements where x23[i] is 1.  When ``M'' is 01, vector operations do not
apply to masked vector elements, and the elements are unchanged in the
register, read on load, and written back on store; when ``M'' is 10, vector
operations do not apply to masked elements, and masked elements are zeroed on
load and written back on store; when ``M'' is 11, vector operations do not
apply to masked elements, loads do not overwrite masked elements in the
registers, and stores do not affect the memory contents targeted by masked
elements.

\begin{table}[htp]
    \begin{small}
        \begin{center}
            \begin{tabular}{cl}
                \hline
                \multicolumn{1}{|c|}{M Register} &
                \multicolumn{1}{c|}{Meaning} \\
                \hline
                \multicolumn{1}{|c|}{00} &
                \multicolumn{1}{c|}{No masking}\\
                \hline
                \multicolumn{1}{|c|}{01} &
                \multicolumn{1}{c|}{Load and store masked elements unchanged}\\
                \hline
                \multicolumn{1}{|c|}{10} &
                \multicolumn{1}{c|}{Load masked elements as zero}\\
                \hline
                \multicolumn{1}{|c|}{11} &
                \multicolumn{1}{c|}{Ignore:  loads and stores do not overwrite masked elements}\\
                \hline
            \end{tabular}
        \end{center}
    \end{small}
    \caption{Accrued exception flag encoding.}
    \label{bitdef}
\end{table}

Vector operations are instructions operating on $rd=vh$, $rs1=vh$, $rs2=vl$,
plus load operations with $rd=vh$ or $rd=vl$.  The operations \inlinecode{add},
\inlinecode{sub}, \inlinecode{sll}, \inlinecode{srl}, \inlinecode{sra},
\inlinecode{and}, \inlinecode{or}, \inlinecode{xor}, \inlinecode{slt},
\inlinecode{slti}, all ``M'' extension instructions, \inlinecode{lr} and
\inlinecode{sc} variants, and floating point operations are valid vector
operations; sub-word-size integer operations

Kerberos executes instructions in the given form by expanding them to many
instructions at the current operating width.  For example, after a
\inlinecode{LH x31, offset(rs)} and with $Size=4$ and $S=1$ on RV64, an $add
x31, x31, x19$ will add each of 16 values each 16-bits wide and packed 4 per
register across $x19$ through $x16$ to the current values in $x31$ through
$x28$, storing in the latter.  Other values of $rd$, $rs1$, and $rs2$ have no
special effect.  If the load were $LW$, it would operate on the two sets of 8
values each 32-bits wide.

This allows Kerberos to load a large amount of data into two sets of registers,
perform large numbers of calculations split across large numbers of execution
units, and write those values back to main memory in single instructions.  When
not holding vector data, writes to any register other than $x31$ (and for loads,
other than $x19$ as well) and operations not between $x31$ and $x19$ with
result stored in $x31$ operate as normal.

All registers below the upper vector range are available for program logic use
immediately after a vector instruction, as well as all between $x19$ and the
lowest upper vector register:  the result is in the upper vector range, and likely
is intended to be written out.  Likewise, because Load and Store only operate
on the upper or lower vector registers, large \inlinecode{memcpy} operations can
use 12 registers at once—96 bytes per instruction for Load and Store, which can
bypass execution units.

Program logic using lightweight vector extensions must test for vector behavior,
possibly by executing a two-register-wide vector \inlinecode{AND} with zero in
$x19$ and $x18$ and with non-zero loaded to $x30$, and then checking if x30 has
changed.  If no change, program behavior must account for no vector extension.

Kerberos achieves this internally by dispatching vector instructions as multiple
regular instructions, tagged for width as normal, with the same \inlinecode{pc}
an an indication of their vector index in the register.  Computed values are
reassembled into full instructions.  \inlinecode{and, or, xor} can operate on
entire registers at once with a single execution unit; \inlinecode{sll, srl,
sra} require an extra cycle to mask the shifted-out bits.

\inlinecode{add, sub} require modified adders able to produce results without
further carry at intermediate stages, or else will occupy many execution units.
These remove several intermediate stages, and all intermediate stages at the
beginning of each word.  A 64-bit adder built in this manner can perform eight
8-bit, four 16-bit, two 32-bit, and one 64-bit addition or subtraction
simultaneously with minimal additional hardware, with the original 64-bit path
being the critical path.

These optimizations allow one vector register to operate on one execution unit,
such that a 12 64-bit register vector addition can produce 96 additions on 12
execution units simultaneously.  Multiple vector instructions can only occur in
parallel in the same pipeline with SMT; on the other hand, a pipeline
provisioned for SMT can make good optimistic use of a non-full pipeline via
vector instructions.

\section{Simultaneous Multi-Threading}

Extending the above, Kerberos can include multiple register files and tracking
data structures on a single pipeline.  $Fetch$ tags each instruction with a
thread ID, and all computations use tables, buffers, and register data
associated with that thread ID.

The operating system can schedule multiple running processes on a single
execution core, with some performance impacts from shared cache resources.
Proper use of Address Space IDs (ASIDs) is critical to keep L1 cache properly
separated.  When only one process is scheduled on a single core, all execution
units are available for use.
