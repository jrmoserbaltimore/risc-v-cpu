\chapter{Prior Soft Processors}

Soft processors are common today, provided as synthesizable hardware definition
language (HDL) code and typically deployed to field programmable gate arrays
(FPGA)—although the term "FPGA" is a misnomer, as the device is a fabric of
complex logic elements and not simple gates.  Modern FPGAs provide hard blocks,
including arithmetic logic units (ALU), memory and peripheral bus controllers,
and even entire hard processor systems (HPS) providing a complete
system-on-chip (SOC).

Soft processors implement custom and specialized low-area designs, historic
video game systems, and even large and complex RISC-V processors.  An
engineering firm can convert a soft processor's HDL design into a hard
processor, achieving higher clock rates and lower power usage; while an FPGA
allows rapid testing of different designs to reduce relative delay and power
usage, either for deployment to FPGA or for eventual translation to ASIC.

Kerberos was inspired by and builds on lessons from three soft processors in
particular:  ZipCPU, iDEA, and Taiga.

\section{ZipCPU}

Gisselquist Technology, LLC maintains a blog\footcite{ZipCPU.Blog} discussing
CPU design and their own ZipCPU.  ZipCPU uses its own instruction set
architecture (ISA), aiming for small size, low power, and fast execution.
Gisselquist licenses ZipCPU under GPLv3, with commercial licensing options.

\section{iDEA}

As is typical of anyone who dedicates time, energy, and passion to a subject,
HuiYan Cheah has produced an enormous amount of research and practical work in
the field of digital design.  Cheah's Ph.D thesis, "The iDEA
Architecture-Focused FPGA Soft Processor," thoroughly documents a soft
processor designed to obtain maximum performance from Xilinx FPGAs.  Like
ZipCPU, iDEA implements a custom ISA, in this case heavily leveraging the
DSP48E1 ALU block provided in Xilinx FPGAs.

Cheah achieved a 453MHz clock rate on both a Xilinx Virtex-6 XC6VLX240T-2 and
an Artix-7.

\section{Taiga}

Matthews and Shannon released Taiga, a RISC-V implementation, under the Apache
License 2.0.  They present a high-level overview of Taiga in two
papers\footcite{Matthews2017}\footcite{Matthews2017b}; little deeper detail is
given about Taiga outside the actual published HDL.
